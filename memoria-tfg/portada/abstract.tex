\cleardoublepage

\chapter*{Abstract\markboth{Abstract}{Abstract}}



%ENGLISH:
%First paragraph (Introduction):
Robotics is a broad field that has been developed with the aim of improving the
quality of life in numerous areas.
There is a wide variety of robots, each dedicated to a specific purpose,
generally matching or surpassing human performance, even working continuously
without the need for rest and eliminating risks.

%Second paragraph (Objectives):
In the educational field, with the growing trend to include robotics in the
curriculum, ROS is the most widely used software, but due to its difficulty, it
is hard to learn, which creates an educational gap in this field.
Another major problem with this software lies in the use of DDS, a communication
protocol that generates a large number of messages, which can saturate the
network.

%Third paragraph (Development Platform):
This work aims to solve these problems by using hardware platforms such as the
Turtlebot 2 and 4 robots, and software tools like Zenoh-Flow, creating a data
flow programming environment compatible with existing ROS2 nodes, making it more
accessible to students in this field.

%Fourth paragraph (Software Architecture with Zenoh and ROS2):
Zenoh-Flow, the main software tool used, was linked with ROS2 by leveraging
Zenoh-bridge-DDS's capability to translate messages bidirectionally between the
Zenoh and DDS protocols.

%Fifth paragraph (Experiments):
The mentioned objectives were achieved through numerous experiments conducted on
an application created following the data flow programming paradigm with
Zenoh-Flow.
In it, several robots must search for and approach an object in an organized
manner, dividing the map equitably and optimizing area sweeping trajectories, so
that the robot that finds the object communicates its position to the rest.

%Sixth paragraph (Conclusions):
This was achieved in simulation and, partially due to external errors, in a real
laboratory environment, demonstrating the viability of data flow programming in
robotics and its compatibility with ROS2.
Its simplicity was also demonstrated, which is an essential requirement for its
application in pre-university robotics education, helping to reduce the
educational gap in this field.
