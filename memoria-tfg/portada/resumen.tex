\cleardoublepage

\chapter*{Resumen\markboth{Resumen}{Resumen}}

%Escribe aquí el resumen del trabajo. Un primer párrafo para dar contexto sobre la temática que rodea al trabajo.\\
%
%Un segundo párrafo concretando el contexto del problema abordado.\\
%
%En el tercer párrafo, comenta cómo has resuelto la problemática descrita en el anterior párrafo.\\
%
%Por último, en este cuarto párrafo, describe cómo han ido los experimentos.


%Primer párrafo (Introducción):
La robótica es un campo amplio que se ha desarrollado con el objetivo de mejorar
la calidad de vida de las personas en numerosos ámbitos.
Existe una gran variedad de robots, cada uno dedicado a un propósito específico,
en el que generalmente igualan o superan el rendimiento humano, incluso
trabajando continuamente sin necesidad de descanso y eliminando riesgos.

%Segundo párrafo (Objetivos):
En el ámbito educativo y con la creciente tendencia a incluir la robótica en el
itinerario formativo, ROS es el software mayormente utilizado y que debido a su
dificultad, resulta difícil de aprender, lo que a su vez genera una brecha en la
educación en este campo.
Otro de los grandes problemas de este software reside en la utilización de DDS,
un protocolo de comunicaciones que genera una gran cantidad de mensajes, que
pueden saturar la red.
%El presente trabajo por tanto tiene como objetivo dar una solución a estos
%problemas, implementando una forma de programación de flujos de datos compatible
%con ROS2.

%En este ámbito y con la creciente tendencia a incluir la robótica en el
%itinerario formativo, surgen grandes vacíos en cuanto al aprendizaje de este
%campo, siendo ROS el \textit{software} mayormente utilizado en robótica, uno de
%los más complejos y dificiles de aprender para alumnos preuniversitarios.
%Este \textit{software} genera una gran congestión de la red debido al uso de
%DDS como protocolo, reduciendo las posibilidades que ofrece la robótica en el
%aprendizaje, sobre todo en cuanto a la colaboración y coordinación en sistemas
%multirobot, campos cada vez más importantes en un mundo en el que los robots
%son cada vez más comunes.

%Tercer párrafo (Plataforma de desarrollo):
El presente trabajo pretende solucionar estos problemas, utilizando plataformas
\textit{hardware} como los robots Turtlebot 2 y 4, y herramientas
\textit{software} como Zenoh-Flow, generando un entorno de programación de
flujos de datos compatible con nodos existentes de ROS2, lo que lo hace más
accesible a los estudiantes de este campo.

%Cuarto párrafo (Arquitectura software con Zenoh y ROS2):
Zenoh-Flow, la principal herramienta \textit{software} utilizada, fue enlazada
con ROS2 aprovechando la capacidad de Zenoh-bridge-DDS para traducir los
mensajes bidireccionalmente entre los protocolos Zenoh y DDS.

%Quinto párrafo (Experimentos):
Los objetivos mencionados fueron alcanzados a través de numerosos experimentos
realizados sobre una aplicación creada siguiendo el paradigma de programación de
flujos de datos con Zenoh-Flow.
En ella, varios robots deben buscar y acercarse a un objeto de manera
organizada, dividiéndose el mapa equitativamente y optimizando trayectorias de
barrido de áreas, de modo que el robot que encuentre dicho objeto, comunique su
posición al resto.

%Sexto párrafo (Conclusiones):
Esto se logró en simulación y, parcialmente debido a errores externos, en un
entorno real de laboratorio, demostrando de esta manera la viabilidad de la
programación de flujos de datos en robótica y su compatibilidad con ROS2.
Asimismo fue demostrada su sencillez, requisito indispensable para su aplicación
en la educación robótica preuniversitaria, ayudando a reducir la brecha
educativa en este ámbito.

