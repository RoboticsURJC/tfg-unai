\chapter{Conclusiones}
\label{cap:capitulo6}

\begin{flushright}
\begin{minipage}[]{10cm}
\emph{El éxito es más un viaje que un destino. El esfuerzo en sí mismo es el premio.}\\
\end{minipage}\\

Arthur Ashe\\
\end{flushright}

\vspace{1cm}

%Escribe aquí un párrafo explicando brevemente lo que vas a contar en este capítulo, que básicamente será una recapitulación de los problemas que has abordado, las soluciones que has prouesto, así como los experimentos llevados a cabo para validarlos. Y con esto, cierras la memoria.

En este último capítulo se detallan los objetivos cumplidos, presentando además
las conclusiones derivadas, así como las habilidades, competencias y
conocimientos adquiridos durante el desarrollo del proyecto, incluyendo futuras
líneas de desarrollo del mismo.

\section{Objetivos cumplidos}
\label{sec:objetivos_cumplidos}

%Enumera los objetivos y cómo los has cumplido.\\
El objetivo principal de este proyecto consiste en desarrollar una forma de
programación y aplicación de flujos de datos en conjunto con ROS2.
\\

El segundo objetivo implica el desarrollo de una aplicación robótica utilizando
la anterior solución para demostrar su viabilidad y funcionamiento.
\\

Ambos objetivos fueron alcanzados, el primero mediante el uso de Zenoh-Flow y
otras tecnologías como Zenoh o Zenoh-bridge-DDS, integradas en conjunto con
ROS2.
Esto dio lugar a un entorno de desarrollo para la programación de aplicaciones en
conjunto con nodos de ROS2.
En cuanto al segundo objetivo, se creó una aplicación robótica para la búsqueda
de objetos, funcional en simulación y parcialmente en un entorno real debido a
limitaciones en el uso de ciertos sensores.
Como resultado, solo parte de la aplicación simulada fue trasladada al entorno
real.
\\

%Enumera también los requisitos implícitos en la consecución de esos objetivos, y cómo se han satisfecho.\\
Como objetivos secundarios se estableció la sencillez de código y comprensión
del entorno para su aplicación en entornos educativos.
Este objetivo ha sido cumplido mediante el requisito de una única clase como
mínimo para la creación de una aplicación robótica simple.
Además, dicha aplicación puede ser programada en Python, un lenguaje conocido
por su naturaleza enfocada principalmente a la sencillez de sintaxis y
programación, siendo una opción popular para el aprendizaje en este ámbito y
proporcionando una sólida base para comprender la programación orientada a
objetos.
\\

Este enfoque también aborda la brecha de aprendizaje en este campo, ya que el
nivel de programación necesario para comprender este entorno es ligeramente
superior a los fundamentos de la programación.
Por lo tanto, podría considerarse como una continuación de una asignatura de
programación o robótica en la educación secundaria, o incluso como una
asignatura independiente, que sienta las bases para comprender ROS2 antes de la
universidad.
\\

En cuanto a los subobjetivos cumplidos podemos detallar los siguientes:

\begin{itemize}
    \item{Se ha desarrollado un entorno de programación que permite la
        programación de flujos de datos en conjunto con nodos de ROS2, haciendo
        posible su comunicación de manera bidireccional, cumpliendo
        simultáneamente los dos primeros subobjetivos marcados.}
    \item{Se ha desarrollado una aplicación robótica en simulación utilizando el
        entorno de programación de flujos de datos mencionado anteriormente.}
    \item{Se ha demostrado la simplicidad de código y la viabilidad de su
        aplicación en entornos educativos, ofreciendo una solución al problema
        de la brecha de aprendizaje.}
\end{itemize}

%Anotación sobre la reducción de tráfico de DDS.
A pesar de haber logrado reducir el tráfico de mensajes originado por DDS
mediante el uso de Zenoh, al seguir utilizando nodos complejos de ROS2 que
generan una gran cantidad de mensajes, esta reducción no ha sido suficiente como
para eliminar por completo el problema de la congestión de red.
\\

%Anotación sobre el traslado de la aplicación a un entorno real.
%En cuanto al objetivo relativo al traslado de la aplicación desarrollada a un
%entorno real, aunque no ha sido cumplido completamente, se ha conseguido
%trasladar la mayoría de nodos de la aplicación, demostrando su viabilidad en
%este entorno, pudiéndose realizar tareas más simples, por ejemplo con motivos
%educativos, de manera completamente funcional.
%\\
Aunque no se ha logrado trasladar completamente la aplicación a un entorno real,
la mayoría de los nodos de la aplicación sí se han trasladado con éxito,
demostrando así su viabilidad en este entorno, por lo que es posible el
desarrollo de tareas más simples de manera completamente funcional, lo que la
hace adecuada para propósitos educativos.
\\

%Anotación sobre los requisitos.
Asimismo se han cumplido todos los requisitos establecidos, como son el uso del
mismo sistema operativo en todas las máquinas (Ubuntu 22.04 LTS), la
compatibilidad de las herramientas \textit{software} utilizadas, la facilidad de
desarrollo y despliegue de las aplicaciones, así como la viabilidad económica
del \textit{hardware} utilizado.
\\

%No olvides dedicar un par de párrafos para hacer un balance global de qué has conseguido, y por qué es un avance respecto a lo que tenías inicialmente. Haz mención expresa de alguna limitación o peculiaridad de tu sistema y por qué es así. Y también, qué has aprendido desarrollando este trabajo.\\


\section{Competencias adquiridas}
\label{sec:competencias_adquiridas}

Además de las competencias numeradas en el Capítulo \ref{cap:capitulo2}, durante
la elaboración de este proyecto, se han obtenido numerosos aprendizajes y
habilidades, entre los que se incluyen los siguientes:

\begin{itemize}
    \item{Conocimiento profundo de algunos aspectos del funcionamiento interno
        de ROS2.}
    \item{Conocimiento profundo del \textit{framework} de Zenoh-Flow, así como
        del protocolo Zenoh y herramientas como Zenoh-bridge-DDS.}
    \item{Conocimiento de las comunicaciones entre ROS2 y Zenoh-Flow, así como
        de las similitudes y distinciones de protocolos como DDS y Zenoh.}
    \item{Descubrimiento de nuevos protocolos con mejores prestaciones como
        Zenoh.}
    \item{Conocimiento más extenso del protocolo SSH y sus métodos de
        autenticación.}
    \item{Conocimiento de nuevos comandos, variables globales, y otros aspectos
        relacionados con el sistema operativo, relativos a ROS2 y a las
        comunicaciones de red.}
    \item{Conocimiento a fondo del paradigma de programación de flujos de
        datos.}
    \item{Adquisición de mayor experiencia, así como el descubrimiento de nuevas
        librerías y funciones relativas a la programación en Python.}
    \item{Conocimiento de nuevos lenguajes de programación como Rust.}
    \item{Conocimiento de herramientas de visualización como PlotJuggler y de
        distintas herramientas de grabación y edición de vídeo.}
    \item{Mayor experiencia con herramientas de creación de imágenes vectoriales
        como Inkscape.}
    \item{Desarrollo de nuevas habilidades y conocimientos respecto a LaTex.}
    \item{Adquisición de conocimientos y posibilidad de tener charlas
        interesantes con otras personas del sector a través de la asistencia y
        presentación del trabajo en ponencias.}
\end{itemize}


\section{Líneas futuras}
\label{sec:lineas_futuras}
%Por último, añade otro par de párrafos de líneas futuras; esto es, cómo se puede continuar tu trabajo para abarcar una solución más amplia, o qué otras ramas de la investigación podrían seguirse partiendo de este trabajo, o cómo se podría mejorar para conseguir una aplicación real de este desarrollo (si es que no se ha llegado a conseguir).

Las herramientas \textit{software} utilizadas relacionadas con Zenoh, como
pueden ser Zenoh-bridge-DDS o el propio Zenoh-Flow, han pasado por muchos
cambios durante la realización de este trabajo debido a su estado de desarrollo,
o incluso de fase beta, como es el caso de este último, por lo que una innegable
línea futura pasa por la actualización de estos \textit{softwares} a versiones
estables, mejorando la experiencia de usuario así como la posible solución de
\textit{bugs} o incompatibilidades.
Además existen nuevos \textit{softwares}, como el reciente RMW de Zenoh, que
podrían ser probados en este entorno, ofreciendo posibles mejoras al mismo.
\\

Además es posible, como segunda línea futura, el empaquetamiento de estas
herramientas \textit{software} en una aplicación que brinde una interfaz
gráfica, así como herramientas interactivas a este entorno, facilitando el
despliegue y la simplicidad de uso en entornos educativos y ahorrando tiempo
tanto a alumnos, como al profesorado.
En esta hipotética aplicación, la instalación y compilación de las herramientas
\textit{software} utilizadas se realizaría automáticamente.
\\
