\chapter{Objetivos}
\label{cap:capitulo2}

\begin{flushright}
\begin{minipage}[]{10cm}
\emph{Dame seis horas para talar un árbol y pasaré las primeras cuatro afilando el hacha}\\
\end{minipage}\\

Abraham Licoln\\
\end{flushright}

\vspace{1cm}

Una vez establecido el marco contextual de este proyecto, se procederá a
presentar una descripción del problema abordado, así como el proceso creativo e
intelectual que ha guiado el desarrollo del mismo, que incluirá los requisitos
del proyecto, la metodología empleada y el plan de trabajo detallado.

%Escribe aquí un párrafo explicando brevemente lo que vas a contar en este capítulo. En este capítulo lo ideal es explicar cuáles han sido los objetivos que te has fijado conseguir con tu trabajo, qué requisitos ha de respetar el resultado final, y cómo lo has llevado a cabo; esto es, cuál ha sido tu plan de trabajo.\\

\section{Descripción del problema}
\label{sec:descripcion}

Este proyecto surge como respuesta a la escasa investigación sobre los flujos de
datos en conjunto con ROS2, ofreciendo a su vez un entorno propicio para la
creación sencilla de aplicaciones, eludiendo la complejidad inherente de este
\textit{middleware} robótico.

Además, tiene como objetivo secundario cerrar la brecha educativa entre la
enseñanza secundaria y universitaria, expuesta en el capítulo anterior.

%Cuenta aquí el objetivo u objetivos generales y, a continuación, concrétalos mediante objetivos específicos.

\section{Requisitos}
\label{sec:requisitos}

Para solucionar lso problemas descritos, este trabajo debe cumplir los
siguientes requisitos:

\begin{enumerate}
    \item{Se utilizará \textit{GNU/Linux}, con la distribución
        \textit{Ubuntu 22.04 LTS} como sistema operativo en todos los
        \textit{hardwares}.}
    \item{Desarrollar alguna forma de programación de flujos de datos con ROS2.}
    \item{El entorno de programación debe brindar la posibilidad de funcionar en
        conjunto con nodos de ROS2, permitiendo la comunicación con los mismos
        mediante \textit{topics}.}
    \item{Los \textit{softwares} utilizados deben ser compatibles para funcionar
        correctamente en conjunto.}
    \item{Las aplicaciones demostrativas desarrolladas deben ser fácilmente
        reproducibles y desplegables tanto en un entorno simulado como en un
        ambiente educativo real o de laboratorio.}
    \item{El desarrollo del \textit{software} debe ser lo suficientemente
        sencillo para poder ser llevado a cabo por alumnos preuniversitarios.}
    \item{El \textit{hardware} utilizado debe ser suficientemente económico para
        ser adquirido por organismos educativos.}
\end{enumerate}

Las competencias generales que se han cumplido, según la guía docente de la
asignatura consisten en:

\begin{enumerate}
    \item{\textit{CB2.} Que los estudiantes sepan aplicar sus conocimientos a su
        trabajo o vocación de una forma profesional y posean las competencias
        que suelen demostrarse por medio de la elaboración y defensa de
        argumentos y la resolución de problemas dentro de su área de estudio.}
    \item{\textit{CB4.} Que los estudiantes puedan transmitir información,
        ideas, problemas y soluciones a un público tanto especializado como no
        especializado.}
    \item{\textit{CB5.} Que los estudiantes hayan desarrollado aquellas
        habilidades de aprendizaje necesarias para emprender estudios
        posteriores con un alto grado de autonomía.}
\end{enumerate}

La competencia específica \textit{CE28} de la asignatura especifica lo
siguiente:
Desarrollo de las capacidades adecuadas para realizar un ejercicio original
individual (o excepcionalmente colectivo), presentarlo y defenderlo ante un
tribunal universitario, consistente en un proyecto en el ámbito de las
tecnologías específicas del campo de la Robótica de naturaleza profesional en el
que se sinteticen e integren las competencias adquiridas en las enseñanzas.


%Describe los requisitos que ha de cumplir tu trabajo.

\section{Metodología}
\label{sec:metodologia}

La metodología utilizada sigue pautas de invesigación sobre el estado del arte
previo al trabajo, y posteriormente sobre el \textit{software} utilizado,
siempre evaluando de antemano la compatibilidad con el \textit{hardware}
disponible, así como realizando pruebas pertinentes sobre su correcto
funcionamiento en los distintos entornos, incluyendo la simulación y el
laboratorio.

A la hora de desarrollar \textit{software} demostrativo se siguió una
metodología de prueba y error debidamente justificada.
Este proceso implicó pruebas periódicas en simulación con el fin de identificar
errores y perfeccionar los valores de los parámetros.
Posteriormente, se evaluó su funcionamiento en un entorno real, como el del
laboratorio.

%Qué paradigma de desarrollo software has seguido para alcanzar tus objetivos.

\section{Plan de trabajo}
\label{sec:plantrabajo}

El desarrollo del proyecto ha comprendido varias etapas, incluyendo la
investigación del \textit{software} a utilizar, la investigación del estado del
arte, la implementación de una arquitectura \textit{software} funcional en los
distintos entornos y el desarrollo del \textit{software} demostrativo o de
ejemplo, comprendiendo un periodo de tiempo superior a un año, comenzando en
Febrero de 2023 y finalizando en Mayo de 2024.

%[TODO] modificar la fecha final.

\begin{enumerate}
    \item{\textit{Investigación del software a utilizar.} Periodo de Febrero a
        Mayo de 2023 durante mis prácticas de empresa, en las que estuve
        aprendiendo el funcionamiento de \textit{softwares} como Zenoh,
        Zenoh-Flow, Zenoh-bridge-DDS, CycloneDDS, y acerca de las
        telecomunicaciones entre robots, 35 horas semanales durante 4 meses.}
    \item{\textit{Investigación del estado del arte.} Periodo de Junio a Agosto
        de 2023, en el que se investigó acerca de los trabajos previos
        relacionados, y sobre la viabilidad y compatibilidad del proyecto.}
    \item{\textit{Implementación de una arquitectura software funcional en
        simulación.} Periodo de Junio a Agosto de 2023, en el que se consiguió
        hacer funcionar el \textit{software} en simulación.}
    \item{\textit{Desarrollo de software demostrativo.} Periodo de Agosto a
        Noviembre de 2023 en el que se migró el \textit{software} desarrollado
        durante las prácticas a versiones posteriores.}
    \item{\textit{Implementación de una arquitectura software funcional en un
        entorno real.} Periodo de Noviembre de 2023 a Enero de 2024 en el que se
        consiguió hacer funcionar el \textit{software} en el laboratorio.}
    \item{\textit{Pruebas del software desarrollado en el laboratorio.} Periodo
        de Noviembre de 2023 a Abril de 2024 en el que se realizaron las pruebas
        y cambios necesarios para hacer funcionar el \textit{software}
        demostrativo en el entorno real del labratorio.}
\end{enumerate}

Durante los periodos de desarrollo de este proyecto fuera de las prácticas de
empresa, se dedcaban aproximadamente de 30 a 40 horas semanales entre las que se
incluyen reuniones con el tutor, que generalmente se llevaban a cabo
semanalemente aunque ocasionalmente cada dos semanas.
Este proceso continuo dió lugar a más de un año de esfuerzo.

El proceso de trabajo ha sido realizado mediante contribuciones a un repositorio
en GitHub\footnote{https://github.com/RoboticsURJC/tfg-unai}, así como su
posterior explicación y desarrollo en el apartado de la
Wiki\footnote{https://github.com/RoboticsURJC/tfg-unai/wiki} del mismo, haciendo
las veces de bitácora, donde quedan reflejados todos los contratiempos,
soluciones y pruebas realizados.

%Qué agenda has seguido. Si has ido manteniendo reuniones semanales, cumplimentando objetivos parciales, si has ido afinando poco a poco un producto final completo, etc.
